\documentclass[journal,12pt,twocolumn]{IEEEtran}
\usepackage{cite}
\usepackage{amsmath,amssymb,amsfonts,amsthm}
\usepackage{algorithmic}
\usepackage{graphicx}
\usepackage{textcomp}
\usepackage{xcolor}
\usepackage{txfonts}
\usepackage{listings}
\usepackage{enumitem}
\usepackage{mathtools}
\usepackage{gensymb}
\usepackage{comment}
\usepackage[breaklinks=true]{hyperref}
\usepackage{tkz-euclide} 
\usepackage{listings}                                   
\def\inputGnumericTable{}                                 
\usepackage[latin1]{inputenc}                                
\usepackage{color}                                            
\usepackage{array}                                            
\usepackage{longtable}                                       
\usepackage{calc}                                             
\usepackage{multirow}                                         
\usepackage{hhline}                                           
\usepackage{ifthen}                                           
\usepackage{lscape}
\newtheorem{theorem}{Theorem}[section]
\newtheorem{problem}{Problem}
\newtheorem{proposition}{Proposition}[section]
\newtheorem{lemma}{Lemma}[section]
\newtheorem{corollary}[theorem]{Corollary}
\newtheorem{example}{Example}[section]
\newtheorem{definition}[problem]{Definition}
\newcommand{\BEQA}{\begin{eqnarray}}
\newcommand{\EEQA}{\end{eqnarray}}
\newcommand{\define}{\stackrel{\triangle}{=}}
\newcommand{\brak}[1]{\langle #1 \rangle}
\theoremstyle{remark}
\newtheorem{rem}{Remark}

\begin{document}
\bibliographystyle{IEEEtran}
\vspace{3cm}
\title{\textbf{10.5.4-3}}
\author{EE23BTECH11023-ABHIGNYA GOGULA}
\maketitle
\newpage
\bigskip
\renewcommand{\thefigure}{\theenumi}
\renewcommand{\thetable}{\theenumi}
\textbf{Question:}
\\
 A ladder has rungs 25cm apart. The rungs decrease uniformly in length from 45cm at the bottom to 25cm at the top. If the top and bottom rungs are 2 and 1/2 meter apart, what is the length of wood required for the rungs?

\begin{table}[h]
\centering
\begin{tabular}{|c|c|}
\hline
Parameter & Value \\
\hline
$n$ & 10 \\
\hline
$x(0)$ & 45 \\
\hline
$x(10)$ & 25 \\
\hline
\end{tabular}
\caption{Description of Parameters}
\label{tab:parameter-values}
\end{table}

\section*{Solution}
Total number of rungs:
\begin{align}
\frac{\brak{\frac{5}{2}}100}{25}+1 = 11
\end{align}

As the length of rungs decreases uniformly, it is in A.P:
\begin{align}
x(n) = \brak{x(0) + nd}u(n) \\
25 = 45 + 10d \\
d = -2
\end{align}

The sum of the lengths of all rungs gives the total length of wood required. So, finding the sum of A.P using Z-transform:
\begin{align}
X(z) = \frac{45}{1-z^{-1}} + \frac{-2z^{-1}}{\brak{1-z^{-1}}^{2}}\\
y(n)=x(n)*u(n)\\
Y(z)=X(z)U(z)\\
Y(z)=\frac{45}{\brak{1-z^{-1}}^2}+ \frac{-2z^{-1}}{\brak{1-z^{-1}}^{3}}
\end{align}
Taking the inverse of the Z-transform using counter-integration:
\begin{align}
y(n) = x(0)\brak{\brak{n+1}u(n)}+\frac{d}{2}\brak{n\brak{n+1}u(n)} \\
y(n) = \frac{n+1}{2}\brak{2x(0)+nd}u(n) \\
y(10) = 385
\end{align}

The length of wood required for the rungs is $385$ cm.
\end{document}

