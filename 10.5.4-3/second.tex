\let\negmedspace\undefined
\let\negthickspace\undefined
\documentclass[journal,12pt,twocolumn]{IEEEtran}
\usepackage{cite}
\usepackage{amsmath,amssymb,amsfonts,amsthm}
\usepackage{algorithmic}
\usepackage{graphicx}
\usepackage{textcomp}
\usepackage{xcolor}
\usepackage{txfonts}
\usepackage{listings}
\usepackage{enumitem}
\usepackage{mathtools}
\usepackage{gensymb}
\usepackage{comment}
\usepackage[breaklinks=true]{hyperref}
\usepackage{tkz-euclide} 
\usepackage{listings}                                   
\def\inputGnumericTable{}                                 
\usepackage[latin1]{inputenc}                                
\usepackage{color}                                            
\usepackage{array}                                            
\usepackage{longtable}                                       
\usepackage{calc}                                             
\usepackage{multirow}                                         
\usepackage{hhline}                                           
\usepackage{ifthen}                                           
\usepackage{lscape}
\newtheorem{theorem}{Theorem}[section]
\newtheorem{problem}{Problem}
\newtheorem{proposition}{Proposition}[section]
\newtheorem{lemma}{Lemma}[section]
\newtheorem{corollary}[theorem]{Corollary}
\newtheorem{example}{Example}[section]
\newtheorem{definition}[problem]{Definition}
\newcommand{\BEQA}{\begin{eqnarray}}
\newcommand{\EEQA}{\end{eqnarray}}
\newcommand{\define}{\stackrel{\triangle}{=}}
\theoremstyle{remark}
\newtheorem{rem}{Remark}

\begin{document}
\bibliographystyle{IEEEtran}
\vspace{3cm}
\title{\textbf{10.5.4-3}}
\author{EE23BTECH11023-ABHIGNYA GOGULA$^{*}$}
\maketitle
\newpage
\bigskip
\renewcommand{\thefigure}{\theenumi}
\renewcommand{\thetable}{\theenumi}
\textbf{Question:}
\\
 A ladder has rungs 25cm apart.The rungs decrease uniformly in length from 45cm at the bottom to 25cm at the top.If the top and bottom rungs are 2 and 1/2 meter apart.what is length of wood required for the rungs?
\\
\textbf{Solution:}
\\
\textbf{To find the total length of wood required for the rungs of the ladder:}

The ladder has rungs uniformly spaced at 25 cm apart. The lengths of the rungs decrease uniformly from 45 cm at the bottom to 25 cm at the top.

\textbf{Calculate the number of rungs:}

The top and bottom rungs are 2 and 1/2 meters apart, which is 250 cm ($2.5$ meters $\times$ $100$ cm/meter $= 250$ cm).
The total length of the ladder: $\text{Total length} = \text{distance between top and bottom rungs} + \text{length of bottom rung} + \text{length of top rung}$
$\text{Total length} = 250 \text{ cm} + 45 \text{ cm} + 25 \text{ cm} = 320 \text{ cm}$

\textbf{Next, calculate the number of rungs:}

$\text{Number of rungs} = \frac{\text{Total length of the ladder}}{\text{Spacing between rungs}}$
$\text{Number of rungs} = \frac{320 \text{ cm}}{25 \text{ cm}} = 12.8$

However, a fraction of a rung isn't possible, so there are 12 rungs in total.

\textbf{Now, calculate the sum of the lengths of the rungs:}

$\text{Sum of rung lengths} = \text{Number of rungs} \times \text{Average of first and last rung length}$
$\text{Sum of rung lengths} = 12 \times \left(\frac{45 \text{ cm} + 25 \text{ cm}}{2}\right) = 12 \times \left(\frac{70 \text{ cm}}{2}\right) = 12 \times 35 \text{ cm} = 420 \text{ cm}$

Therefore, the total length of wood required for the rungs of the ladder is 420 centimeters.

\begin{tabular}{|c|c|}
\hline
\textbf{Parameter} & \textbf{Value} \\
\hline
Spacing between rungs & 25 cm \\
Length of bottom rung & 45 cm \\
Length of top rung & 25 cm \\
Total length of ladder & 320 cm \\
\hline
\end{tabular}


         
\end{document}
